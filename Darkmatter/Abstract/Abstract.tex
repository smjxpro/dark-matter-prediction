
Dark matter, accounting for a large part of the mass in the universe, yet elusive of detection in conventional observations because it does not interact with electromagnetic radiation.\cite{planck2016} Prediction of the properties and dynamics of dark matter is hence a burning issue in astrophysics. In the following, we investigate the use of machine-learning (ML) approaches to predict dark matter properties from cosmological data and theory. Using many different supervised and unsupervised learning algorithms, such as decision trees, support vector machines or deep neural networks, we study a variety of input features including galaxy rotation curves, weak lensing data and cosmic microwave background measurements. Our results show that machine learning models can be used to infer patterns in large datasets which are the signature of dark matter distribution and characteristics, helping to shed light onto its elusive nature. Moreover, our predictive models give encouragingly accurate predictions for the dark matter density profiles and possible interactions with visible matter. This study highlights the power of ML in advancing the fundamental understanding of dark matter, serving as a harbinger for further research and experimental verification.


\vspace{8pt}
\textbf{Keywords:} Dark matter, machine learning, cosmological data, galaxy rotation curves, gravitational lensing, cosmic microwave background, prediction models, supervised learning, unsupervised learning, neural networks, astrophysics, density profiles, dark matter interactions, predictive modeling. 

 

