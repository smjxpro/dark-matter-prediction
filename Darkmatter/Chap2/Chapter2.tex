
\chapter{Literature Review}
\label{chap:2}

\section{Overview} 

 This chapter introduces both dark matter and machine learning (ML) applications for astrophysics, and for dark matter studies in particular. The outline of this chapter is as follows: we first give the theoretical background of dark matter (Sect. 2), the role of machine learning in astrophysics (Sect. 3), and previous works that exploit ML for dark matter studies (Sect. 4). The present article provides a backdrop to the current study in the larger body of extant literature, recognizing opportunities and challenges while searching for gaps in knowledge.

\section{Introduction to Dark Matter}

This section gives a historical and theoretical introduction to the dark matter, as the search takes place in history of discovery and chronological series of developed models. Early theories, including those developed by Fritz Zwicky and Vera Rubin, are discussed, and then the present concept of dark matter as a fundamental feature of the universe is described. This part of the article also reviews as major candidates for dark matter, phenomenon like WIMPs (Weakly Interacting Massive Particles), axions and sterile neutrinos, noting the evidences that motivate their study from observation such as galactic rotation curves, gravitational lensing and CMB (Cosmic microwave background radiation).\cite{bertone2005,jungman1996}

\section{Machine Learning in Astrophysics} 

Machine learning in Astrophysics This section is primarily about how machine learning can be integrated into the field of astrophysical research. It presents the basics of ML and its rapidly growing importance for working with large, complex data sets which are typical in astronomy and cosmology. We compare different classes of machine learning algorithms (supervised, unsupervised and deep learning) as used for other astrophysical applications like star classification, galaxy morphology, cosmological simulations and so forth.\cite{wikiDM}

\section{Current Models for Dark Matter} 

Here we enter the chapter discussing some of those models that have been taken as benchmarks to describe dark matter and its properties. The standard cosmological model, the Lambda Cold Dark Matter (Lambda-CDM) model, is reviewed and its successes in explaining the large-scale structure of the universe are detailed.\cite{peebles1993,dodelson2003} The section also addresses alternative dark matter theories and models such as modified Newtonian dynamics, MOND that contest the Lambda-CDM model.\cite{famaey2012} The shortcomings of the existing models and the necessity for more sophisticated novel ones are also emphasized.

\section{Dark Matter Prediction via Machine Learning} 

In this section we describe some machine learning applied to dark matter analysis. Their applications include predicting the dark matter distributions in galaxy clusters, analyzing the rotation curves of galaxies and refining detection equipment for dark matter candidates such as WIMPs or axions. Past attempts of application of ML algorithms for the search of DM interactions in cosmic and laboratory experiments are subjected to a critical review.\cite{ackermann2015,fermilat2017} It also discusses how ML can be used to refine the dark matter models and enhance the sensitivity of current and future detection techniques.

\section{Quests for Dark Matter} 

In this section, we review all the major problems of dark matter search: the absence of direct detection, contractedness of dark matter theory and experimental methods. It also discusses the challenges of using machine learning in dark matter studies: complex data, overfitting and interpretability. The section will also discuss the ambiguities in DM candidates that hinder progress in the construction of model-dependent ML archetypes for particular kinds of DM.

\section{Summary and Knowledge Gaps}

A Summary of the Findings from Literature and Gaps on What Has Been Known Ultimately, it is important to summarize what are the specific findings and what we do not know about them. "The existence of measurements and accompanying limitations in our understanding thus far are the wellsprings for this dissertation, particularly those related to the use of machine learning methods in predicting dark matter distribution, boosting detection sensitivity, and simulating dark matter interaction." The section will stress the necessity of further research for ML to be able to contribute in advancing our knowledge about dark matter and the transformative possibilities arising from employing astrophysical observations and ML together.
